\documentclass[11pt]{l3deliverable}
\usepackage[table]{xcolor}
\usepackage{geometry}
\usepackage{fullpage}
\usepackage{graphicx}
\usepackage{tabularx}
\usepackage{hyperref}
\usepackage{xcolor}
\newcolumntype{T}{>{\centering\arraybackslash}m{0.20\textwidth}}
\newcolumntype{L}{>{\centering\arraybackslash}m{0.74\textwidth}}
\definecolor{medium-red}{rgb}{0.35,0,0}
\hypersetup{colorlinks, linkcolor={medium-red}, citecolor={medium-red},
urlcolor={medium-red}}
\geometry{a4paper}
\setlength\parindent{0pt}
\title{Acceptance Test Report}
\deliverableID{D8.3}
\project{PSD3 Group Exercise 2}
\team{W}
\version{1.0}
\author{
    Gordon Reid: 1002536R\\
    Ryan Wells: 1002253W\\
    Kristopher Stewart: 1007175S\\
    David Selkirk: 1003646S\\
    James Gallagher: 0800899G\\
}
\date{\today}

\begin{document}

\maketitle

\newpage

\tableofcontents

\newpage

\section{Introduction}

\subsection{Identification}

This is the acceptance test report completed by Team W based on the acceptance
test plan created by Team W and ran on an complete system implementation
of the internship management system supplied by Team V. The internship
management system is for Software Engineering (SE) and Electronic and
Software Engineering (ESE) students, studying in the School of Computing
Science.

\subsection{Related Documentation}

PSD3 Group Exercise Description:

\url{http://fims.moodle.gla.ac.uk/file.php/128/coursework/psd3-ge-2.pdf}

PSD3 Requirements Specification:

\url{fims.moodle.gla.ac.uk/file.php/128/coursework/requirements-specification-ge2.pdf}

\subsection{Purpose and Description Of Document}

This document serves as the Acceptance Test Report based on the test cases
created by Team W, ran on the internship management implementation by Team V.

\newpage

\section{Testing}

\subsection{Approve Accepted Offer}

\begin{tabularx}{\textwidth}{|T|L|}
\hline
Identifier & UtilityAAO1\\
\hline
Date Tested & 18/02/2013\\
\hline
Tester & Gordon Reid - 1002536r\\
\hline
Description & Login as course coordinator and accept a student's internship.\\
\hline
Setup & None possible - unable to add own users or modify them.\\
\hline
Pass/Fail & \cellcolor{red}Fail\\
\hline
Expected Outcome & Student 100000 has their internship approved.\\
\hline
Actual Outcome & Null pointer exception.\\
\hline
Additional Comments & Due to login failure and the facade assumption that a
user is logged in, null pointer exception happens.\\
\hline
\end{tabularx}

\vspace{2em}

\begin{tabularx}{\textwidth}{|T|L|}
\hline
Identifier & UtilityAAO2\\
\hline
Date Tested & 18/02/2013\\
\hline
Tester & Gordon Reid - 1002536r\\
\hline
Description & Login as student and accept a student's internship.\\
\hline
Setup & None possible - unable to add own users or modify them.\\
\hline
Pass/Fail & \cellcolor{red}Fail\\
\hline
Expected Outcome & Student 100000 has their internship still as ACCEPTED.\\
\hline
Actual Outcome & Null pointer exception.\\
\hline
Additional Comments & Due to login failure and the facade assumption that a
user is logged in, null pointer exception happens.\\
\hline
\end{tabularx}

\vspace{2em}

\begin{tabularx}{\textwidth}{|T|L|}
\hline
Identifier & UtilityAAO3\\
\hline
Date Tested & 18/02/2013\\
\hline
Tester & Gordon Reid - 1002536r\\
\hline
Description & Login as course coordinator and accept a student's internship.
The student does not exist.\\
\hline
Setup & None possible - unable to add own users or modify them.\\
\hline
Pass/Fail & \cellcolor{red}Fail\\
\hline
Expected Outcome & No change and program doesn't crash.\\
\hline
Actual Outcome & Null pointer exception.\\
\hline
Additional Comments & Due to login failure and the facade assumption that a
user is logged in, null pointer exception happens.\\
\hline
\end{tabularx}



\subsection{Login}

\begin{tabularx}{\textwidth}{|T|L|}
\hline
Identifier & UtilityTCLogin1\\
\hline
Date Tested & 18/02/2013\\
\hline
Tester & Gordon Reid - 1002536r\\
\hline
Description & Login a valid student. User: 100000, Pass: 1234\\
\hline
Setup & None possible - unable to add own users.\\
\hline
Pass/Fail & \cellcolor{red}Fail\\
\hline
Expected Outcome & Student '100000' logged in, true returned.\\
\hline
Actual Outcome & Student not logged in, false returned.\\
\hline
Additional Comments & Supplied user information appears to have been 
incorrect.\\
\hline
\end{tabularx}

\vspace{2em}

\begin{tabularx}{\textwidth}{|T|L|}
\hline
Identifier & UtilityTCLogin2\\
\hline
Date Tested & 18/02/2013\\
\hline
Tester & Gordon Reid - 1002536r\\
\hline
Description & Login an invalid student. User: 100000, Pass: notmypassword\\
\hline
Setup & None possible - unable to add own users.\\
\hline
Pass/Fail & \cellcolor{green}Pass\\
\hline
Expected Outcome & Student not logged in, false returned.\\
\hline
Actual Outcome & Student not logged in, false returned.\\
\hline
Additional Comments & As expected as there would appear to be no student 
users installed.\\
\hline
\end{tabularx}

\vspace{2em}

\begin{tabularx}{\textwidth}{|T|L|}
\hline
Identifier & UtilityTCLogin3\\
\hline
Date Tested & 18/02/2013\\
\hline
Tester & Gordon Reid - 1002536r\\
\hline
Description & Login an invalid student. User: 0000000a, Pass: notmypassword\\
\hline
Setup & None possible - unable to add own users.\\
\hline
Pass/Fail & \cellcolor{green}Pass\\
\hline
Expected Outcome & Student not logged in, false returned.\\
\hline
Actual Outcome & Student not logged in, false returned.\\
\hline
Additional Comments & As expected as there would appear to be no student 
users installed.\\
\hline
\end{tabularx}

\vspace{2em}

\begin{tabularx}{\textwidth}{|T|L|}
\hline
Identifier & UtilityTCLogin4\\
\hline
Date Tested & 18/02/2013\\
\hline
Tester & Gordon Reid - 1002536r\\
\hline
Description & Login a valid employer. User: empl, Pass: 1234\\
\hline
Setup & None possible - unable to add own users.\\
\hline
Pass/Fail & \cellcolor{red}Fail\\
\hline
Expected Outcome & Employer 'empl' logged in, true returned.\\
\hline
Actual Outcome & Employer not logged in, false returned.\\
\hline
Additional Comments & Supplied user information appears to have been incorrect.
\\
\hline
\end{tabularx}

\vspace{2em}

\begin{tabularx}{\textwidth}{|T|L|}
\hline
Identifier & UtilityTCLogin5\\
\hline
Date Tested & 18/02/2013\\
\hline
Tester & Gordon Reid - 1002536r\\
\hline
Description & Login an invalid employer. User: empl, Pass: notmypassword\\
\hline
Setup & None possible - unable to add own users.\\
\hline
Pass/Fail & \cellcolor{green}Pass\\
\hline
Expected Outcome & Employer not logged in, false returned.\\
\hline
Actual Outcome & Employer not logged in, false returned.\\
\hline
Additional Comments & As expected as there would appear to be no employer users 
installed.\\
\hline
\end{tabularx}

\vspace{2em}

\begin{tabularx}{\textwidth}{|T|L|}
\hline
Identifier & UtilityTCLogin6\\
\hline
Date Tested & 18/02/2013\\
\hline
Tester & Gordon Reid - 1002536r\\
\hline
Description & Login an invalid employer. User: notAnEmployer, Pass: 
notmypassword\\
\hline
Setup & None possible - unable to add own users.\\
\hline
Pass/Fail & \cellcolor{green}Pass\\
\hline
Expected Outcome & Employer not logged in, false returned.\\
\hline
Actual Outcome & Employer not logged in, false returned.\\
\hline
Additional Comments & As expected as there would appear to be no employer users 
installed.\\
\hline
\end{tabularx}

\vspace{2em}

\begin{tabularx}{\textwidth}{|T|L|}
\hline
Identifier & UtilityTCLogin7\\
\hline
Date Tested & 18/02/2013\\
\hline
Tester & Gordon Reid - 1002536r\\
\hline
Description & Login a valid course coordinator. User: tws, Pass: 1234\\
\hline
Setup & None possible - unable to add own users.\\
\hline
Pass/Fail & \cellcolor{green}Pass\\
\hline
Expected Outcome & Course coordinator 'tws' logged in, true returned.\\
\hline
Actual Outcome & Course coordinator 'tws' logged in, true returned.\\
\hline
Additional Comments & Supplied user information appears to have been correct.\\
\hline
\end{tabularx}

\vspace{2em}

\begin{tabularx}{\textwidth}{|T|L|}
\hline
Identifier & UtilityTCLogin8\\
\hline
Date Tested & 18/02/2013\\
\hline
Tester & Gordon Reid - 1002536r\\
\hline
Description & Login an invalid course coordinator. User: tws, Pass: 
notmypassword\\
\hline
Setup & None possible - unable to add own users.\\
\hline
Pass/Fail & \cellcolor{green}Pass\\
\hline
Expected Outcome & Course coordinator not logged in, false returned.\\
\hline
Actual Outcome & Course coordinator not logged in, false returned.\\
\hline
Additional Comments & Supplied user information appears to have been correct.\\
\hline
\end{tabularx}

\vspace{2em}

\begin{tabularx}{\textwidth}{|T|L|}
\hline
Identifier & UtilityTCLogin9\\
\hline
Date Tested & 18/02/2013\\
\hline
Tester & Gordon Reid - 1002536r\\
\hline
Description & Login an invalid course coordinator. User: NotTheCC, Pass: 
notmypassword\\
\hline
Setup & None possible - unable to add own users.\\
\hline
Pass/Fail & \cellcolor{green}Pass\\
\hline
Expected Outcome & Course coordinator not logged in, false returned.\\
\hline
Actual Outcome & Course coordinator not logged in, false returned.\\
\hline
Additional Comments & Supplied user information appears to have been correct.\\
\hline
\end{tabularx}

\newpage

\subsection{Notify Accepted Offer}

\begin{tabularx}{\textwidth}{|T|L|}
\hline
Identifier & UtilityNAO1\\
\hline
Date Tested & 18/02/2013\\
\hline
Tester & Gordon Reid - 1002536r\\
\hline
Description & Login as course coordinator and notify course coordinator of an 
internship.\\
\hline
Setup & None possible - unable to add own users or modify them.\\
\hline
Pass/Fail & \cellcolor{green}Pass\\
\hline
Expected Outcome & Nothing, course coordinators do not do internships.\\
\hline
Actual Outcome & Nothing, course coordinators do not do internships.\\
\hline
Additional Comments &\\
\hline
\end{tabularx}

\vspace{2em}

\begin{tabularx}{\textwidth}{|T|L|}
\hline
Identifier & UtilityNAO2\\
\hline
Date Tested & 18/02/2013\\
\hline
Tester & Gordon Reid - 1002536r\\
\hline
Description & Login as an employer and notify course coordinator of an 
internship.\\
\hline
Setup & None possible - unable to add own users or modify them.\\
\hline
Pass/Fail & \cellcolor{red}Fail\\
\hline
Expected Outcome & Nothing, employers do not do internships.\\
\hline
Actual Outcome & Null pointer exception.\\
\hline
Additional Comments & Due to login failure and the facade assumption that a
user is logged in, null pointer exception happens.\\
\hline
\end{tabularx}

\vspace{2em}

\begin{tabularx}{\textwidth}{|T|L|}
\hline
Identifier & UtilityNAO3\\
\hline
Date Tested & 18/02/2013\\
\hline
Tester & Gordon Reid - 1002536r\\
\hline
Description & Login as course coordinator and accept a student's internship.
The student does not exist.\\
\hline
Setup & None possible - unable to add own users or modify them.\\
\hline
Pass/Fail & \cellcolor{red}Fail\\
\hline
Expected Outcome & No change and program doesn't crash.\\
\hline
Actual Outcome & Null pointer exception.\\
\hline
Additional Comments & Due to login failure and the facade assumption that a
user is logged in, null pointer exception happens.\\
\hline
\end{tabularx}


\newpage

\subsection{Publish Advertisement}

\begin{tabularx}{\textwidth}{|T|L|}
\hline
Identifier & UtilityPA1\\
\hline
Date Tested & 18/02/2013\\
\hline
Tester & Gordon Reid - 1002536r\\
\hline
Description & Login as course coordinator publish an advertisement so a 
student can view it.\\
\hline
Setup & None possible - unable to add own users or modify them.\\
\hline
Pass/Fail & \cellcolor{red}Fail\\
\hline
Expected Outcome & Advertisement is published and available to be viewed by
a student.\\
\hline
Actual Outcome & Null pointer exception.\\
\hline
Additional Comments & Due to login failure and the facade assumption that a user 
is logged in, null pointer exception happens.\\
\hline
\end{tabularx}

\vspace{2em}

\begin{tabularx}{\textwidth}{|T|L|}
\hline
Identifier & UtilityPA2\\
\hline
Date Tested & 18/02/2013\\
\hline
Tester & Gordon Reid - 1002536r\\
\hline
Description & Login as course coordinator publish an advertisement and still
have the course coordinator view it.\\
\hline
Setup & None possible - unable to add own users or modify them.\\
\hline
Pass/Fail & \cellcolor{red}Fail\\
\hline
Expected Outcome & Advertisement is published and available to be viewed by the
course coordinator.\\
\hline
Actual Outcome & Null pointer exception.\\
\hline
Additional Comments & Due to login failure and the facade assumption that a user 
is logged in, null pointer exception happens.\\
\hline
\end{tabularx}

\vspace{2em}

\begin{tabularx}{\textwidth}{|T|L|}
\hline
Identifier & UtilityPA3\\
\hline
Date Tested & 18/02/2013\\
\hline
Tester & Gordon Reid - 1002536r\\
\hline
Description & Login as course coordinator publish an advertisement and still
have the employer who created it view it.\\
\hline
Setup & None possible - unable to add own users or modify them.\\
\hline
Pass/Fail & \cellcolor{red}Fail\\
\hline
Expected Outcome & Advertisement is published and available to be viewed by the
owner employer.\\
\hline
Actual Outcome & Null pointer exception.\\
\hline
Additional Comments & Due to login failure and the facade assumption that a user 
is logged in, null pointer exception happens.\\
\hline
\end{tabularx}

\vspace{2em}

\begin{tabularx}{\textwidth}{|T|L|}
\hline
Identifier & UtilityPA4\\
\hline
Date Tested & 18/02/2013\\
\hline
Tester & Gordon Reid - 1002536r\\
\hline
Description & Login as course coordinator publish an advertisement that
doesn't exist, then try to get a student to view it.\\
\hline
Setup & None possible - unable to add own users or modify them.\\
\hline
Pass/Fail & \cellcolor{red}Fail\\
\hline
Expected Outcome & Nothing happens as advert doesn't exist.\\
\hline
Actual Outcome & Null pointer exception.\\
\hline
Additional Comments & The facade assumes that an advertisement exists so
null pointer exception happens.\\
\hline
\end{tabularx}

\vspace{2em}

\begin{tabularx}{\textwidth}{|T|L|}
\hline
Identifier & UtilityPA5\\
\hline
Date Tested & 18/02/2013\\
\hline
Tester & Gordon Reid - 1002536r\\
\hline
Description & Login as course coordinator publish an advertisement that
doesn't exist, then try to get the course coordinator to view it.\\
\hline
Setup & None possible - unable to add own users or modify them.\\
\hline
Pass/Fail & \cellcolor{red}Fail\\
\hline
Expected Outcome & Nothing happens as advert doesn't exist.\\
\hline
Actual Outcome & Null pointer exception.\\
\hline
Additional Comments & The facade assumes that an advertisement exists so
null pointer exception happens.\\
\hline
\end{tabularx}

\vspace{2em}

\begin{tabularx}{\textwidth}{|T|L|}
\hline
Identifier & UtilityPA6\\
\hline
Date Tested & 18/02/2013\\
\hline
Tester & Gordon Reid - 1002536r\\
\hline
Description & Login as course coordinator publish an advertisement that
doesn't exist, then try to get an employer to view it.\\
\hline
Setup & None possible - unable to add own users or modify them.\\
\hline
Pass/Fail & \cellcolor{red}Fail\\
\hline
Expected Outcome & Nothing happens as advert doesn't exist.\\
\hline
Actual Outcome & Null pointer exception.\\
\hline
Additional Comments & The facade assumes that an advertisement exists so
null pointer exception happens.\\
\hline
\end{tabularx}

\newpage

\subsection{Register Employer}

\begin{tabularx}{\textwidth}{|T|L|}
\hline
Identifier & UtilityRE1\\
\hline
Date Tested & 18/02/2013\\
\hline
Tester & Gordon Reid - 1002536r\\
\hline
Description & Login as course coordinator and add a new employer.\\
\hline
Setup & None possible - unable to add own users or modify them.\\
\hline
Pass/Fail & \cellcolor{green}Pass\\
\hline
Expected Outcome & New employer added to the database.\\
\hline
Actual Outcome & New employer added to the database.\\
\hline
Additional Comments &\\
\hline
\end{tabularx}

\vspace{2em}

\begin{tabularx}{\textwidth}{|T|L|}
\hline
Identifier & UtilityRE2\\
\hline
Date Tested & 18/02/2013\\
\hline
Tester & Gordon Reid - 1002536r\\
\hline
Description & Login as student and add a new employer.\\
\hline
Setup & None possible - unable to add own users or modify them.\\
\hline
Pass/Fail & \cellcolor{red}Fail\\
\hline
Expected Outcome & Nothing, student cannot add employers.\\
\hline
Actual Outcome & Null pointer exception.\\
\hline
Additional Comments & Due to login failure and the facade assumption that a user 
is logged in, null pointer exception happens.\\
\hline
\end{tabularx}

\vspace{2em}

\begin{tabularx}{\textwidth}{|T|L|}
\hline
Identifier & UtilityRE3\\
\hline
Date Tested & 18/02/2013\\
\hline
Tester & Gordon Reid - 1002536r\\
\hline
Description & Login as employer and add a new employer.\\
\hline
Setup & None possible - unable to add own users or modify them.\\
\hline
Pass/Fail & \cellcolor{red}Fail\\
\hline
Expected Outcome & Nothing, employers cannot add employers.\\
\hline
Actual Outcome & Null pointer exception.\\
\hline
Additional Comments & Due to login failure and the facade assumption that a user 
is logged in, null pointer exception happens.\\
\hline
\end{tabularx}

\vspace{2em}


\newpage

\subsection{Submit Advertisement}

\begin{tabularx}{\textwidth}{|T|L|}
\hline
Identifier & UtilitySA1\\
\hline
Date Tested & 18/02/2013\\
\hline
Tester & Gordon Reid - 1002536r\\
\hline
Description & Login as employer and submit an advertisement.\\
\hline
Setup & None possible - unable to add own users or modify them.\\
\hline
Pass/Fail & \cellcolor{red}Fail\\
\hline
Expected Outcome & New advertisement to be published added to the database.\\
\hline
Actual Outcome & Null pointer exception.\\
\hline
Additional Comments & Due to login failure and the facade assumption that a user 
is logged in, null pointer exception happens.\\
\hline
\end{tabularx}

\vspace{2em}

\begin{tabularx}{\textwidth}{|T|L|}
\hline
Identifier & UtilitySA2\\
\hline
Date Tested & 18/02/2013\\
\hline
Tester & Gordon Reid - 1002536r\\
\hline
Description & Login as course coordinator and submit an advertisement.\\
\hline
Setup & None possible - unable to add own users or modify them.\\
\hline
Pass/Fail & \cellcolor{green}Pass\\
\hline
Expected Outcome & New advertisement to be published added to the database.\\
\hline
Actual Outcome & New advertisement to be published added to the database.\\
\hline
Additional Comments &\\
\hline
\end{tabularx}

\vspace{2em}

\begin{tabularx}{\textwidth}{|T|L|}
\hline
Identifier & UtilitySA3\\
\hline
Date Tested & 18/02/2013\\
\hline
Tester & Gordon Reid - 1002536r\\
\hline
Description & Login as student and submit an advertisement.\\
\hline
Setup & None possible - unable to add own users or modify them.\\
\hline
Pass/Fail & \cellcolor{red}Fail\\
\hline
Expected Outcome & Nothing, students cannot submit advertisements.\\
\hline
Actual Outcome & Null pointer exception.\\
\hline
Additional Comments & Due to login failure and the facade assumption that a user 
is logged in, null pointer exception happens.\\
\hline
\end{tabularx}

\vspace{2em}

\begin{tabularx}{\textwidth}{|T|L|}
\hline
Identifier & UtilitySA4\\
\hline
Date Tested & 18/02/2013\\
\hline
Tester & Gordon Reid - 1002536r\\
\hline
Description & Don't login and submit an advertisement.\\
\hline
Setup & None possible - unable to add own users or modify them.\\
\hline
Pass/Fail & \cellcolor{red}Fail\\
\hline
Expected Outcome & Nothing, you need to login to submit advertisements.\\
\hline
Actual Outcome & Null pointer exception.\\
\hline
Additional Comments & Due to the facade assumption that a user 
is logged in, null pointer exception happens.\\
\hline
\end{tabularx}

\newpage

\subsection{View Advertisement}

\begin{tabularx}{\textwidth}{|T|L|}
\hline
Identifier & UtilityVA1\\
\hline
Date Tested & 18/02/2013\\
\hline
Tester & Gordon Reid - 1002536r\\
\hline
Description & Login as student and view a published advertisement.\\
\hline
Setup & None possible - unable to add own users or modify them.\\
\hline
Pass/Fail & \cellcolor{red}Fail\\
\hline
Expected Outcome & Advertisement details available.\\
\hline
Actual Outcome & Null pointer exception.\\
\hline
Additional Comments & Advert creation and publication doesn't work causing
failure.\\
\hline
\end{tabularx}

\vspace{2em}

\begin{tabularx}{\textwidth}{|T|L|}
\hline
Identifier & UtilityVA2\\
\hline
Date Tested & 18/02/2013\\
\hline
Tester & Gordon Reid - 1002536r\\
\hline
Description & Login as course coordinator and view a published advertisement.\\
\hline
Setup & None possible - unable to add own users or modify them.\\
\hline
Pass/Fail & \cellcolor{red}Fail\\
\hline
Expected Outcome & Advertisement details available.\\
\hline
Actual Outcome & Null pointer exception.\\
\hline
Additional Comments & Advert creation and publication doesn't work causing
failure.\\
\hline
\end{tabularx}

\vspace{2em}

\begin{tabularx}{\textwidth}{|T|L|}
\hline
Identifier & UtilityVA3\\
\hline
Date Tested & 18/02/2013\\
\hline
Tester & Gordon Reid - 1002536r\\
\hline
Description & Login as owner company and view their published advertisement.\\
\hline
Setup & None possible - unable to add own users or modify them.\\
\hline
Pass/Fail & \cellcolor{red}Fail\\
\hline
Expected Outcome & Advertisement details available.\\
\hline
Actual Outcome & Null pointer exception.\\
\hline
Additional Comments & Advert creation and publication doesn't work causing
failure.\\
\hline
\end{tabularx}

\vspace{2em}

\begin{tabularx}{\textwidth}{|T|L|}
\hline
Identifier & UtilityVA4\\
\hline
Date Tested & 18/02/2013\\
\hline
Tester & Gordon Reid - 1002536r\\
\hline
Description & Login as an employer and view another employers published 
advertisement.\\
\hline
Setup & None possible - unable to add own users or modify them.\\
\hline
Pass/Fail & \cellcolor{red}Fail\\
\hline
Expected Outcome & Advertisement details unavailable.\\
\hline
Actual Outcome & Null pointer exception.\\
\hline
Additional Comments & Advert creation and publication doesn't work causing
failure.\\
\hline
\end{tabularx}

\vspace{2em}

\begin{tabularx}{\textwidth}{|T|L|}
\hline
Identifier & UtilityVA5\\
\hline
Date Tested & 18/02/2013\\
\hline
Tester & Gordon Reid - 1002536r\\
\hline
Description & Login as course coordinator and view a pending advertisement.\\
\hline
Setup & None possible - unable to add own users or modify them.\\
\hline
Pass/Fail & \cellcolor{red}Fail\\
\hline
Expected Outcome & Advertisement details available.\\
\hline
Actual Outcome & Null pointer exception.\\
\hline
Additional Comments & Advert creation and publication doesn't work causing
failure.\\
\hline
\end{tabularx}

\vspace{2em}

\begin{tabularx}{\textwidth}{|T|L|}
\hline
Identifier & UtilityVA6\\
\hline
Date Tested & 18/02/2013\\
\hline
Tester & Gordon Reid - 1002536r\\
\hline
Description & Login as student and view a pending advertisement.\\
\hline
Setup & None possible - unable to add own users or modify them.\\
\hline
Pass/Fail & \cellcolor{red}Fail\\
\hline
Expected Outcome & Advertisement details unavailable.\\
\hline
Actual Outcome & Null pointer exception.\\
\hline
Additional Comments & Advert creation and publication doesn't work causing
failure.\\
\hline
\end{tabularx}

\vspace{2em}

\begin{tabularx}{\textwidth}{|T|L|}
\hline
Identifier & UtilityVA7\\
\hline
Date Tested & 18/02/2013\\
\hline
Tester & Gordon Reid - 1002536r\\
\hline
Description & Login as owner company and view their pending advertisement.\\
\hline
Setup & None possible - unable to add own users or modify them.\\
\hline
Pass/Fail & \cellcolor{red}Fail\\
\hline
Expected Outcome & Advertisement details available.\\
\hline
Actual Outcome & Null pointer exception.\\
\hline
Additional Comments & Advert creation and publication doesn't work causing
failure.\\
\hline
\end{tabularx}

\vspace{2em}

\begin{tabularx}{\textwidth}{|T|L|}
\hline
Identifier & UtilityVA8\\
\hline
Date Tested & 18/02/2013\\
\hline
Tester & Gordon Reid - 1002536r\\
\hline
Description & Login as some employer and view another's pending advertisement.\\
\hline
Setup & None possible - unable to add own users or modify them.\\
\hline
Pass/Fail & \cellcolor{red}Fail\\
\hline
Expected Outcome & Advertisement details available.\\
\hline
Actual Outcome & Null pointer exception.\\
\hline
Additional Comments & Advert creation and publication doesn't work causing
failure.\\
\hline
\end{tabularx}

\newpage

\subsection{View Student Detail}

\begin{tabularx}{\textwidth}{|T|L|}
\hline
Identifier & UtilityVS1\\
\hline
Date Tested & 18/02/2013\\
\hline
Tester & Gordon Reid - 1002536r\\
\hline
Description & Login as course coordinator and view a student.\\
\hline
Setup & None possible - unable to add own users or modify them.\\
\hline
Pass/Fail & \cellcolor{red}Fail\\
\hline
Expected Outcome & Student details available.\\
\hline
Actual Outcome & Null pointer exception.\\
\hline
Additional Comments & No students exist in system and none can be added.\\
\hline
\end{tabularx}

\vspace{2em}

\begin{tabularx}{\textwidth}{|T|L|}
\hline
Identifier & UtilityVS2\\
\hline
Date Tested & 18/02/2013\\
\hline
Tester & Gordon Reid - 1002536r\\
\hline
Description & Login as course coordinator and view a nonexistent student.\\
\hline
Setup & None possible - unable to add own users or modify them.\\
\hline
Pass/Fail & \cellcolor{green}Pass\\
\hline
Expected Outcome & Nothing, student doesn't exist.\\
\hline
Actual Outcome & Nothing, student doesn't exist.\\
\hline
Additional Comments & No students exist in system and none can be added.\\
\hline
\end{tabularx}

\vspace{2em}

\begin{tabularx}{\textwidth}{|T|L|}
\hline
Identifier & UtilityVS3\\
\hline
Date Tested & 18/02/2013\\
\hline
Tester & Gordon Reid - 1002536r\\
\hline
Description & Login as employer and view a student.\\
\hline
Setup & None possible - unable to add own users or modify them.\\
\hline
Pass/Fail & \cellcolor{red}Fail\\
\hline
Expected Outcome & Student details unavailable, employers cannot view
students.\\
\hline
Actual Outcome & Null pointer exception.\\
\hline
Additional Comments & No students exist in system and none can be added.\\
\hline
\end{tabularx}

\vspace{2em}
\begin{tabularx}{\textwidth}{|T|L|}
\hline
Identifier & UtilityVS4\\
\hline
Date Tested & 18/02/2013\\
\hline
Tester & Gordon Reid - 1002536r\\
\hline
Description & Login as student and view a student.\\
\hline
Setup & None possible - unable to add own users or modify them.\\
\hline
Pass/Fail & \cellcolor{red}Fail\\
\hline
Expected Outcome & Student details unavailable, students cannot view
students.\\
\hline
Actual Outcome & Null pointer exception.\\
\hline
Additional Comments & No students exist in system and none can be added.\\
\hline
\end{tabularx}

\vspace{2em}


\newpage

\section{Summary}

\subsection{Supplied Data}

The implementation supplied was incomplete with assignAcademicVisitor(),
recordVisitAssessment(), and createNewSelfSourcedRole(). These methods however
are not necessarily required for the must have use cases.

The implementation has no compiled JavaDoc and, after inspection, the
implementation lacks JavaDoc entirely.

There was a large block comment specifying usernames contained within the
system however only the course coordinator information was supplied. Secondly,
the team's tests used a different user name for student (100000 instead of
stu) however neither successfully login.

There was also no build directory and thus no INSTALL.txt.

In addition, there was no README.txt to know how to contact the team. The
team looked out to the third year computing science Facebook group to contact
the team. This was successful however the team themselves were not entirely
sure how the implementation worked.

These issues meant that modifying our test set up was impossible to conduct
while maintaining the black box testing benefits.

\subsection{Overall test results}

The entire test suite comprised of forty test cases. Of those there were two
failures and twenty seven errors (all java.lang.NullPointerException).

The test cases which passed are exhaustively listed below:

\begin{enumerate}
	\item Student login with incorrect password.
	\item Student login with incorrect username.
	\item Employer login with incorrect password.
	\item Employer login with incorrect username.
	\item Course Coordinator login with correct credentials.
	\item Course Coordinator with incorrect password.
	\item Course Coordinator with incorrect username.
	\item Course Coordinator notifying they have accepted an internship.
	\item Course Coordinator submits an advertisement.
	\item Course Coordinator tries to view details of a nonexistent student.
	\item Course Coordinator registers a new employer.
\end{enumerate}

\subsection{Failed Use Cases}

\subsubsection{Approve Accepted Offer}

The approve accepted offer fails because login fails. Login fails because the
supplied data for student and employers appears to be incorrect and there is
no way to set up a user.

\subsubsection{Login}

The login use case failed for students and employers. As we were only able to
use the supplied data, the failing cannot be entirely verified. One can only
assume that the supplied data for students and employers is incorrect.

\subsubsection{Notify Accepted Offer}

The notify accepted offer fails because login fails. Login fails because the
supplied data for student and employers appears to be incorrect and there is
no way to set up a user.

\subsubsection{Publish Advertisement}

The publish advertisement use case fails because login fails. Login fails 
because the supplied data for student and employers appears to be incorrect and 
there is no way to set up a user.

In addition to this there is an assumption that the advertisement to be
published exists. If publishing a non-existent advertisement is attempted the
application crashes from a null pointer exception.

\subsubsection{Register Employer}

The publish advertisement use case fails because login fails. Login fails 
because the supplied data for student and employers appears to be incorrect and 
there is no way to set up a user.

\subsubsection{Submit Advertisement}

The submit advertisement use case fails because login fails. Login fails 
because the supplied data for student and employers appears to be incorrect and 
there is no way to set up a user.

\subsubsection{View Advertisement}

The submit advertisement use case fails because login fails. Login fails 
because the supplied data for student and employers appears to be incorrect and 
there is no way to set up a user.

In addition advert publication failing also causes the use case to fail.

\subsubsection{View Student Detail}

The view student use case fails because there are no students in the system and
none can be added.

\end{document}
